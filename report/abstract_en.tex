\chapter*{Abstract}
A great deal of effort in model-based testing is related to the creation of the model. In addition, the model itself, while a powerful tool of abstraction, can have conceptual errors, introduced by the tester. These problems can be reduced by generating those models automatically. This dissertation presents a dynamic reverse engineering approach that aims to extract part of the model of an existing Web application through the identification of User Interface (UI) patterns. This reverse engineering approach explores automatically any Web application, records information related to the interaction, analyzes the gathered information, tokenizes it, and infers the existing UI patterns via syntactical analyzing. After complemented with additional information and validated, the model extracted is the input for the Pattern-Based Graphical User Interface Testing (PBGT) approach for testing existing web application under analysis.

First the theme developed during the course of the dissertation is introduced, starting by defining the context and issue at hand and describing the goals of this dissertation. Afterwards we present a literary review on reverse engineering approaches, approaches that infer patterns from Web applications, and approaches that extract models from applications. We explain the approach in detail, present a case study, and lastly, we present the conclusions taken from the work.

\chapter*{Resumo}
\begin{otherlanguage}{portuguese}
Grande parte do esforço despendido em testes baseados num modelo está relacionado com a criação do modelo. Além disso, o modelo, mesmo sendo uma ferramenta poderosa de abstração, pode ter erros conceptuais introduzidos pelo testador. Esses problemas podem ser reduzidos ao gerar esses modelos automaticamente. Esta dissertação apresenta uma abordagem de engenharia reversa dinâmica que pretende extrair parte do modelo de uma aplicação Web a testar, diretamente da sua interface gráfica, através da identificação de padrões de interface de utilizador (\textit{UI Patterns}). Esta abordagem de engenharia reversa explora automaticamente qualquer aplicação Web, guarda informação relacionada com a interação, analisa a informação guardada, analisa-a lexicalmente, e infere padrões através de análise sintáctica. Após ser complementado com informação adicional e validado, o modelo resultante serve de entrada para a abordagem de Teste de GUIs baseado em Padrões (PBGT), para testar a aplicação Web sobre análise.

Primeiro o tema desenvolvido no decorrer da dissertação é introduzido, começando por definir o seu contexto, motivação e objectivos. Depois é apresentada uma revisão bibliográfica de abordagens que usam engenharia reversa, abordagens que inferem padrões de aplicações Web, e abordagens que extraem modelos de aplicações.  A abordagem desenvolvida é apresentada em detalhe, é apresentado um caso de estudo, e finalmente são apresentadas as conclusões do trabalho realizado.
\end{otherlanguage}