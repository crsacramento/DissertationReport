\chapter{Introduction} \label{chap:intro}

\section*{}

\section{Context} \label{sec:context}

Web applications are getting more and more important. Due to their stability and security against losing data, there is a growing trend to move applications towards the Web, with the most notorious examples being Google's mail and office software applications. Web applications can now handle tasks that before could only be performed by desktop applications \cite{garrett2005ajax}, like editing images or creating spreadsheet documents.

Despite the relevance that Web applications have in the community, they still suffer from a lack of standards and conventions \cite{constantine2002usage}, unlike desktop and mobile applications. This means that the same task can be implemented in many different ways, which makes automated testing difficult to accomplish and inhibits reuse of testing code. For instance, authentication (\textit{login}) failure usually triggers the appearance of an error message, but some implementations simply erase the inserted data, with no error message visible.

GUIs \textit{(Graphical User Interfaces)} of all kinds are populated with recurring behaviors that vary slightly; examples being authentication via \textit{login/password} pair and content search. These behaviors (patterns) are called User Interface (UI) patterns \cite{van2001patterns} and are recurring solutions to common design problems. Due to their widespread use, UI patterns allow users a sense of familiarity and comfort when using applications.

However, while UI patterns are familiar to users, their implementation may vary significantly. Despite this, it is possible to define generic and reusable test strategies (User Interface Test Patterns - UITP) to test those patterns. This requires a configuration process, in order to adapt the tests to different applications \cite{dalal1999model}.

Testing is an increasingly important part of any development process as it is essential to improve the trust in the quality of software. Thus, there is an ever increasing need to develop new techniques in this field. This necessity is increased by the constant improvement of development techniques.

When it comes to GUI testing, there are some applicable testing techniques \cite{memon2002gui}: \textbf{capture replay}, which requires a functional GUI; \textbf{unit testing}, which requires the manual implementation of the tests and may involve too much work in order to test all of the functionalities; \textbf{random input testing}, which is good at finding situations where the system crashes, but is not as good at finding other kinds of errors; and \textbf{model-based testing}, which enables automatic test case generation and execution, even though it requires a formal model in order to generate the test cases.

\section{Motivation and Objectives} \label{sec:goals}

As mentioned before, the focus of this dissertation is a component of a research project named PBGT (\textit{Pattern-based GUI Testing}) \cite{moreira2013pattern}. The main goal of of this project is to improve current model-based GUI testing methods and tools, contributing to conceive an effectively applicable testing approach in industry and to contribute to the construction of higher quality GUIs and software systems. One of the problems to overcome when implementing a model-based GUI testing approach is the time required to construct the model and the test case explosion problem. Choosing the right abstraction level of the model, extracting part of that model by a reverse engineering process and focusing the test cases to cover common recurrent behavior seems to be a way to solve these problems.

The proposal aims to continue the work done on PARADIGM-RE, a component of the PBGT process responsible for extracting part of the Web application model from the Web application itself via reverse engineering, by developing a completely new tool to extract a partial model of a Web application via reverse engineering.

The previous tool \cite{nabuco2013inferring,nabuco2014inferring} is a dynamic reverse engineering approach that extracts User Interface (UI) Patterns from existent Web applications. Its functioning can be summed up like this: 

\begin{itemize}
\item First, the user interacts with the Web application, navigating it to the best of his ability while the tool records information related to the interaction (user actions and parameters, and for each page visited, its HTML source and URL);
\item Second, the collected information is analyzed in order to calculate several metrics (e.g., the differences between subsequent HTML pages);
\item Lastly, the existent UI Patterns are inferred from the overall information calculated based on a set of heuristic rules.
\end{itemize}

This approach was evaluated on several widely used Web applications and the results were deemed satisfactory, since the tool identified most of the occurring patterns and their location on the page. However, there are some patterns the tool doesn't identify, such as the Menu pattern, and the heuristics are considered to be still in an incipient state. The tool's major weakness is that it requires a user to interact with a Web application, a process that can quickly become morose and time-consuming.

As stated before, this dissertation aims to develop a new tool for extracting a partial model from a Web aplication meant to be tested. The major goals for this dissertation were to improve the existing reverse engineering and pattern inferring process, remove the need for user interaction, and automate the model construction; or in other words, independently/automatically explore a Web application, infer the existing UI patterns in its pages, and finally produce a model with the UI Test Patterns that define the strategies to test the UI Patterns present in the web application. This is meant to speed up model construction, and mitigate the number of errors introduced into the model.

\section{Structure of the Report} \label{sec:outline}

This document is structured into four main chapters. In this first section, Chapter \ref{chap:intro}, we start by introducing the theme to be developed during the course of the dissertation, define the context, motivations, goals of this dissertation and the issue at hand. Chapter \ref{chap:sota} introduces essential concepts to understand the problems with which this document deals, and presents the state of the art on reverse engineering. Chapter \ref{chap:approach} presents the developed tool, RE-TOOL, focusing on its architecture and functionalities and on some of the challenges which needed to be tackled during the development. Chapter \ref{chap:validation} presents a case study. Finally, Chapter \ref{chap:conclusion} presents some conclusions about this research work, along with the limitations of the implementation.